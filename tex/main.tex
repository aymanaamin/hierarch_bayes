% !TeX spellcheck = en_US
\documentclass[a4paper,fleqn,11pt]{article}

% header
\include{header}

\begin{document}

% title page
%
% header

\pagenumbering{Roman}

\title{\huge Forecasting Swiss Exports using Bayesian Forecast Reconciliation}

\author[$\dagger$]{Florian Eckert}
\author[$\ddagger$]{Rob J. Hyndman}
\author[$\ddagger$]{Anastasios Panagiotelis}
\affil[$\dagger$]{\small KOF Swiss Economic Institute, ETH Zurich}
\affil[$\ddagger$]{\small Department of Econometrics and Business Statistics, Monash University}
\date{Preliminary Version: \today}

\maketitle
\begin{abstract}
This paper conducts an extensive forecasting study on 13,118 time series measuring Swiss exports grouped by export destination and product category.  Due to the fact that some series are aggregates of others, to ensure that economic agents make aligned decisions, it is critical that forecasts cohere with known constraints.  To achieve this we apply existing state of the art methods in forecast reconciliation.  In addition, we also introduce a novel Bayesian reconciliation framework. The Bayesian approach allows for explicit estimation of reconciliation biases, leading to several innovations: It is possible to use prior judgment to assign weights to specific forecasts and the occurrence of negative reconciled forecasts can be ruled out. Overall we find strong evidence that in addition to producing coherent forecasts, forecast reconciliation also leads to improvements in forecast accuracy.  Forecast reconciliation also outperform alternative approaches such as the simple aggregation of bottom level forecasts.\\

\noindent \textbf{JEL Classification:} C32, C53, E17 \\
\noindent \textbf{Keywords}: Hierarchical Forecasting, Aggregation, Regularization, Forecast Reconciliation, Optimal Forecast Combination.
\end{abstract}
\clearpage




% table of contents
%\tableofcontents

%\clearpage

%\listoffigures
%\listoftables

%\clearpage

\pagenumbering{arabic}

			




% 2. Model	
\section{Introduction}
\label{sec:model}

Example of a hierarchy with $k = 3$ levels, $m = 13$ series in total and $q = 9$ series at the bottom of the hierarchy.\\

\begin{figure}[H]
	\centering
	\begin{forest}
	before packing={
		forked edges,
	}
		[{$Y_0$}
			 [{$Y_{01}$}
		 		[{$Y_{011}$}]
				[{$Y_{012}$}]
		 		[{$Y_{012}$}]
		 	]
			 [{$Y_{02}$}
		 		[{$Y_{021}$}]
				[{$Y_{022}$}]
		 		[{$Y_{023}$}]
		 	]
			 [{$Y_{03}$}
				[{$Y_{031}$}]
				[{$Y_{032}$}]
				[{$Y_{033}$}]
			]
		]
	\end{forest}
\vspace{0.5cm}
	\caption{Simple Example Hierarchy}
\end{figure}

If we define $Y_t = [Y_0, Y_{01}, Y_{02}, \hdots, Y_{033}]'$ to be an ($m \times 1$) vector of top-down stacked observations from all levels, it must hold at each point in time $t$ that
\begin{align}
Y_t &= S Y_{k,t}
\end{align}
where $Y_{k,t}$ is a ($q \times 1$) vector containing the observations at level $k$, the bottom of the hierarchy, and $S$ is an ($m \times q$) aggregation matrix. For the above example, $S$ has to be constructed in the following way.
\begin{align*} S &=
\begin{bmatrix}
\ 1 & 1 & 1 & 1 & 1 & 1 & 1 & 1 & 1\ \ \\
\ 1 & 1 & 1 & 0 & 0 & 0 & 0 & 0 & 0\ \ \\
\ 0 & 0 & 0 & 1 & 1 & 1 & 0 & 0 & 0\ \ \\
\ 0 & 0 & 0 & 0 & 0 & 0 & 1 & 1 & 1\ \ \\
\ 1 & 0 & 0 & 0 & 0 & 0 & 0 & 0 & 0\ \ \\
\ 0 & 1 & 0 & 0 & 0 & 0 & 0 & 0 & 0\ \ \\
  &   &   &   & \vdots &   &   &   &   \\
\ 0 & 0 & 0 & 0 & 0 & 0 & 0 & 0 & 1\ \ 
\end{bmatrix}
\end{align*}
For the forecasts to be consistent and additive, they usually have to be estimated either top-down or bottom-up. However, there are benefits to forecasting all series in the hierarchy. It might be that different models provide better fits, varying information sets are available, or judgment predictions have to be used. As a result of this, the different levels of the forecasted hierarchy usually cannot be aggregated consistently. In order to reconcile forecasts at each level of the hierarchy, \cite{Hyndman2011} show that optimal predictions at the bottom level can be obtained using the following regression approach.
\begin{align}
Y_t(h) &= S\beta_{h} + e_t(h)
\end{align}
where $Y_t(h)$ is an ($m \times 1$) vector containing the h-periods-ahead forecasts at time $t$ for each level in the hierarchy. $\beta_{h}$ represents the optimal predictions at the bottom level that minimize the deviations between the individual forecasts and the consistent hierarchy. The reconciliation error term $e_t(h)$ follows a normal distribution with mean 0 and covariance matrix $\Sigma_h$. There is obviously a high level of heteroskedasticity in the error terms, which is why $\beta_h$ has to be estimated using generalized least squares. The optimal point forecasts result therefore from the following weighted least squares regression.
\begin{align}
\label{eq:reg}
\beta_{h} &= \left(S'\Sigma_h^{-1}S \right)^{-1} S'\Sigma_h^{-1}Y_t(h)
\end{align}
Intuitively, $\beta_h$ minimizes the reconciliation error, which is the squared distance between the actual and reconciled forecasts. As a result of the weighting matrix, forecasts with a higher prediction error receive less weight in the regression.\\


\section{Bayesian Forecast Reconciliation}
\subsection{Proposition}
Since every forecast horizon is reconciled independently, the time subscripts are dropped from now on to simplify notation. It is assumed that we have $n$ samples from the predictive distribution of each of the $m$ predictions. Therefore, $Y_{i}$ denotes a vector length $m$ that contains a draw $i$ from the predictive distribution of all forecasts, where $i = 1,\hdots, n$. The unobservable error term consists of two components, a prediction error $e_{i}$ and a fixed effect $\alpha$. The latter is a vector of length $m$ and can be interpreted as containing the reconciliation errors that are specific to each forecasted variable. In other terms, $\alpha$ is the difference between the unreconciled forecast mean $\hat{Y}$ and the reconciled forecast mean $\tilde{Y}$. Furthermore, there is no common intercept by design. When all predictors are zero, no bottom-level forecasts are aggregated and the expected response should therefore be zero as well. Because the slope line passes through the origin, the reconciliation errors also do not necessarily sum to zero. The following regression equation can then be used to estimate $\alpha$ and $\beta$.
\begin{align}
Y_{i} &= \alpha + S\beta + e_{i}
\end{align}
where $e$ follows a normal distribution with mean zero and variance $\Sigma$. An error-components model of this form is quite frequently used in panel data regressions. Since the aggregation matrix $S$ is fixed, the reconciliation error is uncorrelated with the explanatory variables. It is therefore possible to treat the reconciliation errors as random effects and to omit $\alpha$ from the above regression. The optimal forecasts at the bottom level are then still estimated consistently. 

Estimating the reconciliation errors explicitly as a fixed effect has several advantages, but proves to be rather tricky. A standard technique in panel models is to get rid of the fixed effect by demeaning the data or taking first differences. If there is a sufficient number of observations, the fixed effects can then be retrieved by subtracting the fitted values from the responses. This is not possible here because there is no variation in the regressor $S$. Another approach would be to include a dummy for each variable to account for the fixed effects. However, this is unfeasible too because it leads to perfect multicollinearity in the explanatory variables.

\begin{figure}[H]
\label{jacksonpollock}
\begin{tikzpicture}[
    thick,
    >=stealth',
    dot/.style = {
      draw,
      fill = white,
      circle,
      inner sep = 0pt,
      minimum size = 4pt
    }
  ]
    \coordinate (O) at (0,0);
    \draw (0,0)+(0:0.9) arc(0:45:0.9);
    \node at (22.5:0.6) {\footnotesize $45^\circ$};
    \draw[->] (0,0) -- (0,7) coordinate[label = {below left:$Y_{i}$}] (ymax);
    \draw[->] (0,0) -- (10,0) coordinate[label = {below left:$S \beta$}] (xmax);
    \draw[gray] (0,0) -- (6,6);
    \foreach \Point in {(2,1.4), (2,1.9), (2,2), (2,2.1), (2,2.3),
                      (2,2.7), (2,2.8), (2,3.2), (2,3.4), (2,3.5)} \node at \Point {\textbullet};
    \foreach \Point in {(3,2.5), (3,2.8), (3,3), (3,3.2), (3,3.3),
                      (3,3.7), (3,3.8), (3,4), (3,4.1), (3,4.4)} \node at \Point {\textbullet};
    \foreach \Point in {(5,2.8), (5,3.1), (5,3.4), (5,3.7), (5,3.8),
                      (5,4.2), (5,4.3), (5,4.5), (5,4.9), (5,5.4)} \node at \Point {\textbullet};    

    \node [label = {[shift={(-0.35,-0.55)}] \footnotesize $\hat{Y}_T$}] at (5,4) {$\circ$};
    \node [label = {[shift={(-0.3,-0.55)}] \footnotesize $\hat{Y}_2$}] at (3,3.5) {$\circ$};
    \node [label = {[shift={(-0.3,-0.55)}] \footnotesize $\hat{Y}_1$}] at (2,2.5) {$\circ$};
    
    \draw[gray,dashed] (0,5) node[left]{\footnotesize $\tilde{Y}_T = \tilde{Y}_1 + \tilde{Y}_2$} -- (5,5);
    \draw[gray,dashed] (0,3) node[left]{\footnotesize $\tilde{Y}_2$} -- (3,3);    
    \draw[gray,dashed] (0,2) node[left]{\footnotesize $\tilde{Y}_1$} -- (2,2); 

    \draw[gray,decoration={brace,raise=5pt},decorate]
  (5.2,5) -- node[right=6pt] {\footnotesize Reconciliation error for $Y_T$} (5.2,4);
    
\end{tikzpicture}
\caption{Graphical Representation of a Reconciliation Regression.}
\end{figure}

A convenient solution to this identification problem comes from Bayesian econometrics. There are several other fields, where identification is achieved by means of informative priors. In structural vector auto-regressions, the structural coefficient matrices can be identified through prior assumptions about the sign of shocks \citep{Baumeister2015}. In dynamic factor models, different rotations of latent factors and loadings are observationally equivalent. The factors can be identified through prior restrictions on the factor loadings \citep{Bai2015}.

In Bayesian statistics, data is considered to be fixed and parameters are treated as random variables. The researcher has a prior belief about the distribution of the parameters in a model. After observing the data, this prior belief is combined with the likelihood according to Bayes' theorem in order to obtain the posterior distribution of the parameters. This principle can be applied to $\alpha$, $\beta$ and $\Sigma$ in the reconciliation regression. The consistent bottom-level forecasts $\beta$ follow a normal distribution with mean $b$ and covariance $B$. The reconciliation errors $\alpha$ are also distributed normally with mean $a$ and covariance $A$. The prediction errors follow a normal distribution with mean zero and variance $\Sigma$. The aggregation matrix $S$ and the sample of unreconciled forecasts $Y$ are believed to be fixed.
\begin{align}
f(\alpha, \beta, \Sigma\ |\ S,Y) \propto f(Y,S\ |\ \alpha, \beta, \Sigma) \times f(\alpha, \beta, \Sigma)
\end{align}
In other words, the posterior distribution of the bottom-level forecasts is proportional to the likelihood (of the hierarchy to be consistent) times the prior distribution of the bottom-level forecasts. We have little prior knowledge about the reconciled bottom-level forecasts $\beta$ and the predictive distribution $\Sigma$. However, we may have some prior belief about the reconciliation errors because we trust some forecasting models more than others. This could be due to better data availability, better forecasting performance in the past or subjective judgment of the forecaster. A sufficiently narrow prior for $\alpha$ allows the identification of the model parameters.\\




\subsection{Estimation}
In order to approximate the distribution of $Y_t(h)$, we draw $n$ h-periods-ahead-predictions $Y_i$, with $i = 1, \hdots, n$, from their forecasted distributions. This is equivalent to solving a seemingly unrelated regression using generalized least squares. Following \cite{Greenberg2008}, the likelihood function for the data is given by 
\begin{align*}
f(Y,S\ |\ \alpha,\beta,\Sigma) \propto \frac{1}{|\Sigma|}\exp\left[\frac{1}{2} \sum_i (Y_i - \alpha - S\beta)'\Sigma^{-1}(Y_i - \alpha - S\beta)\right]
\end{align*}
and the posterior distribution is accordingly given by
\begin{align*}
f(\alpha,\beta,\Sigma\ |\ Y,S) & \propto \frac{1}{|\Sigma|^{n/2}}\exp\left[-\frac{1}{2} \sum_i (Y_i - \alpha - S\beta)'\Sigma^{-1}(Y_i - \alpha - S\beta)\right] \\
&\times \exp \left[-\frac{1}{2}(\alpha - a_0)'A_0^{-1}(\alpha - a_0)\right] \\
&\times \exp \left[-\frac{1}{2}(\beta - b_0)'B_0^{-1}(\beta - b_0)\right] \\
&\times \frac{1}{|\Sigma|(v_0 - m - 1)} \exp \left[-\frac{1}{2} tr(R_0^{-1}\Sigma^{-1}) \right]
\end{align*}
We are interested in the marginal distribution of $\alpha$, $\beta$ and $\Sigma$. Following \cite{Percy1992}, this can be achieved by approximating the joint posterior distribution through Gibbs sampling from the conditional distributions and then averaging the samples. First some random starting values for the parameters are chosen. Afterwards, the following steps are repeated until convergence:

\begin{enumerate}
\item \textbf{Draw $\alpha$ conditional on $\beta,\Sigma,Y,S$}\\
The fixed effects $\alpha$ can be obtained from a regression of $Y_i - S\beta$ on an identity matrix. The conditional posterior distribution for $\alpha$ is given by
\begin{align}
	\label{eq:alpha}
\alpha\ |\ \beta,\Sigma,Y,S &\sim N(a_1,A_1)
\end{align}
where
\begin{align*}
A_1 &= \left(\sum_i \Sigma^{-1} + A_0^{-1}\right)^{-1} \\
a_1 &= A_1 \left(\sum_i \Sigma^{-1} (Y_i - S\beta) + A_0^{-1}a_0\right)
\end{align*}
The prior mean $a_0$ should be set to zero and the prior covariance $A_0$ should be a diagonal matrix. The entries on the diagonal should be lower for forecasts that the researcher is more confident in. Entries that correspond to uncertain forecasts can be set higher. Perhaps a reasonable prior can be derived from the level of aggregation that the forecast is in. Different weighting schemes might be tested and conclusions can be drawn from the forecast combination literature, such as \cite{Cesur2016} or \cite{Brooks2001}.\\

\item \textbf{Draw $\beta$ conditional on $\alpha,\Sigma,Y,S$}\\
The slope coefficients $\beta$ can be obtained from a regression of $Y_i - \alpha$ on $S$. The conditional posterior distribution for $\beta$ is given by
\begin{align}
\beta\ |\ \alpha,\Sigma,Y,S &\sim N(b_1,B_1)
\end{align}
where
\begin{align*}
B_1 &= \left(\sum_i S'\Sigma^{-1}S + B_0^{-1}\right)^{-1} \\
b_1 &= B_1 \left(\sum_i S'\Sigma^{-1} (Y_i - \alpha) + B_0^{-1}b_0\right)
\end{align*}
Unless we have some reason to believe otherwise, the priors $b_0$ and $B_0$ should be chosen as uninformative as possible. Prior belief on the divergence between the initial and the reconciled forecasts should be incorporated in the prior on $\alpha$. \\

\item \textbf{Draw $\Sigma$ conditional on $\alpha,\beta,Y,S$}\\
$\Sigma$ is the covariance matrix of the prediction errors. Depending on how the draws from each predictive distribution are ordered in $Y$, there is more or less structure in the off-diagonal elements. $\Sigma$ can be drawn from an inverse Wishart distribution.
\begin{align}
\Sigma\ |\ \alpha,\beta,Y,S \sim W^{-1}(v_1,R_1)
\end{align}
where
\begin{align*}
v_1 &= v_0 + n\\
R_1 &=  \left( R_0^{-1} + \sum_i (Y_i - \alpha - S \beta)'(Y_i - \alpha - S \beta) \right)^{-1}
\end{align*}

\end{enumerate}

\noindent Convergence of the Gibbs sampler can be checked using the recursive mean of the chains. After convergence is achieved, a sufficiently large number of these draws are saved. The mean of the hierarchically consistent forecasts can then be computed using discrete statistics on the saved draws from the posterior.\\

\subsection{Prior Selection}
As noted before, the priors for the parameters should be as diffuse as possible with the exception of $A_0$, the prior variance of the reconciliation errors $\alpha$. $A_0$ can also reflect prior knowledge about the reconciliation errors. A high value on the diagonal of $A_0$ implies a belief that the reconciliation errors are larger, whereas a small value implies a belief that the reconciliation error should be small. There are several outcomes, depending on the information content of the prior.
\begin{itemize}
    \item\textbf{Diffuse prior}. Setting the diagonal entries on the prior variance $A_0$ close to infinity assumes that we have no knowledge about the reconciliation errors. As a result, the parameters cannot be uniquely identified and the Gibbs sampler does not converge.
    \item \textbf{Strong prior}. Setting the diagonal entries on the prior variance $A_0$ very close to zero assumes that the reconciliation errors are distributed very tight around the prior mean, which is zero. In this case, the information in the prior dominates the informational content of the data entirely and $\beta$ reduces to the generalized least squares estimate in \cite{Hyndman2016}. 
    \item \textbf{Informed prior}. Parameter shrinkage has been studied extensively in the literature on regularization techniques for high-dimensional data, where penalties lead to the selection of models with fewer predictors and less variance. Well known examples include the Lasso and the Horseshoe penalty functions.\footnote{See for instance \cite{Piironen2017} for a comprehensive overview.}
    
    However, our prior knowledge about the reconciliation errors requires not only certain parameters to be shrunk towards zero, but others to be less restricted. A suitable framework is given by the global-local scale mixtures of normal distributions by \cite{Polson2010}. They assume that the prior variance of the coefficient $\alpha_j$ consists of a global variance component $\tau^2$ and a local variance component $\lambda_j^2$. A commonly used technique in this framework is the adaptive ridge estimator developed by \cite{Brown1980}, which has been shown by \cite{Firinguetti1997} and \cite{Haitovsky1987} to take an equivalent form to equation (\ref{eq:alpha}) in a seemingly unrelated regression framework. This implies that $A_0$ can be interpreted as a ridge matrix. Following \cite{Polson2010}, an appropriate form of the adaptive ridge matrix in a global-local shrinkage framework is given by
    \begin{align*}
	A_0 &= \tau^{2}(\lambda \Sigma \lambda')
    \end{align*}
    Using $\tau^2 = 1/n$ as the global variance component and $\lambda_j^2 = 1$ for each local variance component shrinks the posterior variance of $\alpha$ towards the squared standard errors in an intercept only regression. This leads to the familiar GLS result in a hierarchical reconciliation.
    
    Deviations from this global shrinkage can be controlled by letting the local components deviate from unity. However, if we want to shrink some of the reconciliation errors to zero, we want at the same time to increase the variation of the remaining elements in $\alpha$ such that they are able to capture the increased aggregation error at their level of the hierarchy. One way to do this is to keep the total dispersion of the reconciliation errors constant. A convenient measure for total variability in a multivariate normal distribution is the generalized variance described in \cite{Mustonen1997}. It is defined as the determinant of a covariance matrix. Since our baseline local variance component $\lambda^2$ is an identity matrix, its diagonal elements can be changed in any way as long as the determinant of $\lambda^2$ remains constant at 1. A simple approach is the following scaling function:
    \begin{align*}
   	|\lambda^2| &= \lambda_1^2 \lambda_2^2 \hdots \lambda_m^2 
   	= \prod_{s^- = 1}^{x} \lambda_{s^-}\ \eta^{-\frac{1}{x}}   \prod_{s^+ = x+1}^{m} \lambda_{s^+}\ \eta^{\frac{1}{m-x}} = 1
    \end{align*}
    The $x$ local variance components $\lambda_{s^-}$ are scaled down by a factor $\eta^{\frac{1}{x}}$ and the remaining $(m-x)$ components $\lambda_{s^+}$ are correspondingly scaled up by a factor $\eta^{\frac{1}{m-x}}$. The generalized variance remains at unity irrespective of the scaling factor $\eta$. Treating $A_0$ as a random parameter, we draw it from an inverse Wishart distribution conditional on the local variance hyperparameters $\lambda^2$. 
    \begin{align}
    	A_0\ |\ \alpha,\beta,Y,S,\lambda^2 \sim W^{-1}\left(n,\lambda \Sigma \lambda'\right)
    \end{align}  
    This setup has several advantages: The prior variance for the reconciliation errors is informed by the data and depends on the other model parameters. This allows us to place very few restrictions on the model and at the same time selectively shrink certain reconciled forecast towards their base forecasts.\footnote{The adaptive ridge estimator is therefore given by $\hat{\alpha} = \Bigl(n \Sigma^{-1} + n (\lambda \Sigma \lambda')^{-1}\Bigr)^{-1} \Bigl(\sum_i \Sigma^{-1} (Y_i - S\beta)\Bigr)$}
   
   
\end{itemize}

\clearpage

\section{Simulation Exercise}

Simulate hierarchies of various sizes, levels and time series properties and compare benchmark performance / forecasting accuracy with other methods such as
\begin{itemize}
    \item Top-Down Approach
    \item Bottom-Up Approach
    \item Middle-Out Approach
    \item GLS Reconciliation \cite{Hyndman2011}
    \item Minimum Trace Reconciliation \cite{Wickramasuriya2015}
    \item Alternative Weighting Matrices
    \item No Reconciliation\\
\end{itemize}

What happens if we have forecasts that do not follow a normal distribution? An example from my own experience: Swiss imports are driven quite heavily by imports of airplanes and it is uncertain when they arrive, e.g. forecasts are bimodal.

Another advantage of the bayesian approach: we can discard draws where a bottom level forecast is negative.

\clearpage

\section{Empirical Application}
\subsection{Data}
We use a comprehensive dataset of Swiss foreign trade in goods. For both exports and imports, it contains the nominal value in Swiss francs of goods traded with 245 countries and dependent territories. They are aggreg The goods are categorized according to the economic sector, following a national nomenclature covering 14 main groups and 272 subgroups. The hierarchy is unbalanced, meaning that certain goods categories are available more disaggregated than others. This results in a hierarchy with up to 5 levels. The 14 main groups are the following:
\begin{enumerate}[itemsep=-1ex,partopsep=1ex,parsep=1ex]
    \item Forestry and agricultural products, fisheries
    \item Energy source
    \item Textiles, clothing, shoes
    \item Paper, stationery and graphical products
    \item Leather, rubber, plastics 
    \item Products of the chemical and pharmaceutical industry
    \item Stones and earth
    \item Metals
    \item Machines, appliances, electronics
    \item Vehicles
    \item Precision instruments, clocks and watches and jewellery  
    \item Various goods such as music instruments, home furnishings, toys, sports equipment
    \item Precious metals, precious and semi-precious stones
    \item Works of art and antiques
\end{enumerate}
Because of the geographical and the categorical dimension, there is no unique hierarchical structure. Following \cite{Hyndman2016}, these time series can therefore be thought of as a grouped time series. There are 63'516 time series with non-zero entries in total, 35'602 for the export hierarchy and 27'914 for the import hierarchy. The time series are available in monthly frequency from 1989 on and are not adjusted for working days or seasonality. The data is collected by the Swiss Federal Customs Administration\footnote{\url{https://www.ezv.admin.ch/ezv/en/home/topics/swiss-foreign-trade-statistics.html}} and made available in a machine-friendly data format only on basis of a subscription.\\


\subsection{Results}
Repeat comparison from simulation exercise, construct export and import hierarchies by geographical region, countries and product categories. As a by-product, there could a visualization in the spirit of the MIT Trade Atlas\footnote{\url{https://atlas.media.mit.edu/en/visualize/tree_map/hs92/export/ltu/all/show/2016/ }} be developed.\\

Results from running hts on the data, using the average across the root mean squared errors and mean absolute percentage errors of all series.\\
\begin{table}[H]
\centering
\caption{Forecast Accuracy by Standard Aggregation Methods}
\small
\begin{tabularx}{\textwidth}{Xcclcclcclcc}
\toprule
& \multicolumn{2}{c}{Overall} & & \multicolumn{2}{c}{2003-2007} & & \multicolumn{2}{c}{2008-2012} & & \multicolumn{2}{c}{2013-2018}\\
\cmidrule{2-3} \cmidrule{5-6} \cmidrule{8-9} \cmidrule{11-12}
& \scriptsize{MAPE} & \scriptsize{RMSE} & & \scriptsize{MAPE} & \scriptsize{RMSE} & & \scriptsize{MAPE} & \scriptsize{RMSE} & & \scriptsize{MAPE} & \scriptsize{RMSE}\\ 
\midrule
Bottom-up &  &  &&  & && & && & \\ 
\quad \scriptsize{ETS} & 2.3 & 2.3  && 2.3 & 2.3 && 2.3 & 2.3 && 2.3 & 2.3\\ 
\quad \scriptsize{ARIMA} & 2.3 & 2.3  && 2.3 & 2.3 && 2.3 & 2.3 && 2.3 & 2.3\\
\addlinespace
Middle-out &  &  &&  & && & && & \\ 
\quad \scriptsize{ETS} & 2.3 & 2.3  && 2.3 & 2.3 && 2.3 & 2.3 && 2.3 & 2.3\\ 
\quad \scriptsize{ARIMA} & 2.3 & 2.3  && 2.3 & 2.3 && 2.3 & 2.3 && 2.3 & 2.3\\
\addlinespace
\multicolumn{3}{l}{Top-down (Gross-Sohl A)} &&  & && & && & \\ 
\quad \scriptsize{ETS} & 2.3 & 2.3  && 2.3 & 2.3 && 2.3 & 2.3 && 2.3 & 2.3\\ 
\quad \scriptsize{ARIMA} & 2.3 & 2.3  && 2.3 & 2.3 && 2.3 & 2.3 && 2.3 & 2.3\\
\addlinespace
\multicolumn{3}{l}{Top-down (Gross-Sohl F)} &&  & && & && & \\ 
\quad \scriptsize{ETS} & 2.3 & 2.3  && 2.3 & 2.3 && 2.3 & 2.3 && 2.3 & 2.3\\ 
\quad \scriptsize{ARIMA} & 2.3 & 2.3  && 2.3 & 2.3 && 2.3 & 2.3 && 2.3 & 2.3\\
\addlinespace
\multicolumn{3}{l}{Top-down (Forecast Proportions)} &&  & && & && & \\ 
\quad \scriptsize{ETS} & 2.3 & 2.3  && 2.3 & 2.3 && 2.3 & 2.3 && 2.3 & 2.3\\ 
\quad \scriptsize{ARIMA} & 2.3 & 2.3  && 2.3 & 2.3 && 2.3 & 2.3 && 2.3 & 2.3\\
\bottomrule
\end{tabularx}
\end{table}

Using optimal forecast combination with different weighting schemes:\\
\begin{table}[H]
\centering
\caption{Forecast Accuracy by Optimal Forecast Combination Weights}
\small
\begin{tabularx}{\textwidth}{Xcclcclcclcc}
\toprule
& \multicolumn{2}{c}{Overall} & & \multicolumn{2}{c}{2003-2007} & & \multicolumn{2}{c}{2008-2012} & & \multicolumn{2}{c}{2013-2018}\\
\cmidrule{2-3} \cmidrule{5-6} \cmidrule{8-9} \cmidrule{11-12}
& \scriptsize{MAPE} & \scriptsize{RMSE} & & \scriptsize{MAPE} & \scriptsize{RMSE} & & \scriptsize{MAPE} & \scriptsize{RMSE} & & \scriptsize{MAPE} & \scriptsize{RMSE}\\ 
\midrule
\multicolumn{3}{l}{OLS (unweighted combination) } &&  & && & && & \\ 
\quad \scriptsize{ETS} & 2.3 & 2.3  && 2.3 & 2.3 && 2.3 & 2.3 && 2.3 & 2.3\\ 
\quad \scriptsize{ARIMA} & 2.3 & 2.3  && 2.3 & 2.3 && 2.3 & 2.3 && 2.3 & 2.3\\
\addlinespace
\multicolumn{3}{l}{WLS (forecast variance weights) } &&  & && & && & \\ 
\quad \scriptsize{ETS} & 2.3 & 2.3  && 2.3 & 2.3 && 2.3 & 2.3 && 2.3 & 2.3\\ 
\quad \scriptsize{ARIMA} & 2.3 & 2.3  && 2.3 & 2.3 && 2.3 & 2.3 && 2.3 & 2.3\\
\addlinespace
\multicolumn{3}{l}{MinT (full covariance weights) } &&  & && & && & \\ 
\quad \scriptsize{ETS} & 2.3 & 2.3  && 2.3 & 2.3 && 2.3 & 2.3 && 2.3 & 2.3\\ 
\quad \scriptsize{ARIMA} & 2.3 & 2.3  && 2.3 & 2.3 && 2.3 & 2.3 && 2.3 & 2.3\\
\addlinespace
\multicolumn{3}{l}{nseries (numer of series at each node)} &&  & && & && & \\ 
\quad \scriptsize{ETS} & 2.3 & 2.3  && 2.3 & 2.3 && 2.3 & 2.3 && 2.3 & 2.3\\ 
\quad \scriptsize{ARIMA} & 2.3 & 2.3  && 2.3 & 2.3 && 2.3 & 2.3 && 2.3 & 2.3\\
\bottomrule
\end{tabularx}
\end{table}

\ \\

Results from Bayesian estimation.. In addition, it might be interesting to look at forecast errors for different regional/categorical aggregates:
\begin{table}[H]
\centering
\caption{Forecast Accuracy by Regional and Categorical Aggregates}
\small
\begin{tabularx}{\textwidth}{Xcclcclcclcc}
\toprule
& \multicolumn{2}{c}{Overall} & & \multicolumn{2}{c}{2003-2007} & & \multicolumn{2}{c}{2008-2012} & & \multicolumn{2}{c}{2013-2018}\\
\cmidrule{2-3} \cmidrule{5-6} \cmidrule{8-9} \cmidrule{11-12}
& \scriptsize{MAPE} & \scriptsize{RMSE} & & \scriptsize{MAPE} & \scriptsize{RMSE} & & \scriptsize{MAPE} & \scriptsize{RMSE} & & \scriptsize{MAPE} & \scriptsize{RMSE}\\ 
\midrule
\multicolumn{3}{l}{OLS (unweighted combination) } &&  & && & && & \\ 
\quad \scriptsize{ETS} & 2.3 & 2.3  && 2.3 & 2.3 && 2.3 & 2.3 && 2.3 & 2.3\\ 
\quad \scriptsize{ARIMA} & 2.3 & 2.3  && 2.3 & 2.3 && 2.3 & 2.3 && 2.3 & 2.3\\ 
\bottomrule
\end{tabularx}
\end{table}



\clearpage

\section{Conclusion}







\clearpage

% bibliography
\pagenumbering{Roman}
\setcounter{page}{3}
\bibliography{library}
\bibliographystyle{apalike}

\clearpage


% appendix

\appendix
\section{Appendix}
\subsection{Data}

% table with goods categories
\begin{small}
\begin{longtable}{p{3cm}p{11cm}}
\caption{Tariff Numbers and Descriptions of Goods}\\
\toprule
\normalsize{Tariff Number} & \normalsize{Description}\\
\midrule
\endfirsthead
\multicolumn{2}{@{}l}{\ldots continued}\\
\toprule
\normalsize{Tariff Number} & \normalsize{Description}\\  
\midrule
\endhead
\bottomrule
\multicolumn{2}{r@{}}{continued \ldots}\\
\endfoot
\bottomrule
\endlastfoot
	01	&	Forestry and agricultural products, fisheries	\\
\enskip	01.1	&	Food, beverages and tobacco	\\
\enskip\enskip	01.1.1	&	Basic materials for the food industry	\\
\enskip\enskip\enskip	01.1.1.01	&	Plant-based basic materials for the food industry	\\
\enskip\enskip\enskip	01.1.1.02	&	Animal-based basic materials for the food industry	\\
\enskip\enskip	01.1.2	&	Foods which are ready to consume	\\
\enskip\enskip\enskip	01.1.2.01	&	Fresh foods which are ready to consume	\\
\enskip\enskip\enskip\enskip	01.1.2.01.01	&	Fresh, plant-based foods which are ready to consume	\\
\enskip\enskip\enskip\enskip	01.1.2.01.02	&	Fresh, animal-based foods which are ready to consume	\\
\enskip\enskip\enskip	01.1.2.02	&	Prepared or processed foods	\\
\enskip\enskip\enskip\enskip	01.1.2.02.01	&	Plant-based prepared or processed foods	\\
\enskip\enskip\enskip\enskip	01.1.2.02.02	&	Animal-based prepared or processed foods	\\
\enskip\enskip	01.1.3	&	Beverages	\\
\enskip\enskip\enskip	01.1.3.01	&	Alcohol-free beverages	\\
\enskip\enskip\enskip	01.1.3.02	&	Alcoholic beverages	\\
\enskip\enskip	01.1.4	&	Tobacco	\\
\enskip\enskip\enskip	01.1.4.01	&	Raw tobacco	\\
\enskip\enskip\enskip	01.1.4.02	&	Manufactured tobacco	\\
\enskip	01.2	&	Feeding stuffs for animals	\\
\enskip	01.3	&	Live animals	\\
\enskip	01.4	&	Horticultural products	\\
\enskip	01.5	&	Forestry products (not firewood)	\\
\enskip\enskip	01.5.1	&	Wood and wood products (not firewood)	\\
\enskip\enskip\enskip	01.5.1.01	&	Raw timber (not firewood)	\\
\enskip\enskip\enskip	01.5.1.02	&	Semi-manufactures made out of wood (not firewood)	\\
\enskip\enskip\enskip	01.5.1.03	&	Finished products made out of wood (not firewood)	\\
\enskip\enskip	01.5.2	&	Cork and articles made from cork	\\
\enskip	01.6	&	Products for commercial/industrial further processing such as oils, fats, starches, plants and vegetable parts, etc.	\\
\midrule
	02	&	Energy source	\\
\enskip	02.1	&	Solid combustibles	\\
\enskip\enskip	02.1.1	&	Coal	\\
\enskip\enskip	02.1.2	&	Firewood	\\
\enskip	02.2	&	Petroleum and distillates	\\
\enskip\enskip	02.2.1	&	Crude oil	\\
\enskip\enskip	02.2.2	&	Petrol	\\
\enskip\enskip	02.2.3	&	Diesel oil	\\
\enskip\enskip	02.2.4	&	Heating oil	\\
\enskip\enskip	02.2.5	&	Mineral oils and distillates, other than crude oil, petrol, diesel oil and heating oil	\\
\enskip\enskip	02.2.6	&	Lubricants	\\
\enskip	02.3	&	Gas	\\
\enskip	02.4	&	Electrical energy	\\
\midrule
	03	&	Textiles, clothing, shoes	\\
\enskip	03.1	&	Textiles	\\
\enskip\enskip	03.1.1	&	Basic materials for the textile industry	\\
\enskip\enskip\enskip	03.1.1.01	&	Textile fibres	\\
\enskip\enskip\enskip\enskip	03.1.1.01.01	&	Silk fibres	\\
\enskip\enskip\enskip\enskip	03.1.1.01.02	&	Textile fibres made from wool and other animal hair	\\
\enskip\enskip\enskip\enskip	03.1.1.01.03	&	Cotton fibres	\\
\enskip\enskip\enskip\enskip	03.1.1.01.04	&	Textile fibres made from natural fibres (excluding silk, wool and cotton wool)	\\
\enskip\enskip\enskip\enskip	03.1.1.01.05	&	Textile fibres made from artificial and synthetic fibres	\\
\enskip\enskip\enskip	03.1.1.02	&	Yarns	\\
\enskip\enskip\enskip\enskip	03.1.1.02.01	&	Silk yarns	\\
\enskip\enskip\enskip\enskip	03.1.1.02.02	&	Yarns made from wool and other animal hair	\\
\enskip\enskip\enskip\enskip	03.1.1.02.03	&	Cotton yarns	\\
\enskip\enskip\enskip\enskip	03.1.1.02.04	&	Artificial and synthetic yarns	\\
\enskip\enskip\enskip\enskip	03.1.1.02.05	&	Yarns from textile fibres (except from silk, wool and cotton, as well as artificial and synthetic yarns)	\\
\enskip\enskip\enskip	03.1.1.03	&	Woven fabrics and knitted fabrics	\\
\enskip\enskip\enskip\enskip	03.1.1.03.01	&	Woven fabrics of silk	\\
\enskip\enskip\enskip\enskip	03.1.1.03.02	&	Woven fabrics of wool or of other fine animal hair	\\
\enskip\enskip\enskip\enskip	03.1.1.03.03	&	Woven fabrics of cotton	\\
\enskip\enskip\enskip\enskip	03.1.1.03.04	&	Woven of artificial and synthetic fabrics	\\
\enskip\enskip\enskip\enskip	03.1.1.03.05	&	Woven fabrics made from textiles (excluding those made from silk, wool and cotton, and man-made woven fabrics)	\\
\enskip\enskip\enskip\enskip	03.1.1.03.06	&	Knitted and crocheted fabrics	\\
\enskip\enskip	03.1.2	&	Special textile fabrics	\\
\enskip\enskip\enskip	03.1.2.01	&	Pile fabrics	\\
\enskip\enskip\enskip	03.1.2.02	&	Embroidery	\\
\enskip\enskip\enskip	03.1.2.03	&	Gauze, tulles and lace	\\
\enskip\enskip	03.1.3	&	Home textiles	\\
\enskip\enskip\enskip	03.1.3.01	&	Carpets and other textile floor coverings, linoleum	\\
\enskip\enskip\enskip	03.1.3.02	&	Bedding textiles and household textiles	\\
\enskip\enskip\enskip	03.1.3.03	&	Curtains and non-embroidered decorative articles such as wallpaper, wall coverings and drapes	\\
\enskip\enskip	03.1.4	&	Textiles for technical uses and other products such as wadding, felt, nonwovens, special yarns, cordage, narrow woven fabrics and covered or impregnated fibres	\\
\enskip\enskip\enskip	03.1.4.01	&	Nonwovens	\\
\enskip\enskip\enskip	03.1.4.02	&	Special yarns, narrow woven fabrics, labels, wicks, etc.	\\
\enskip\enskip\enskip	03.1.4.03	&	Covered or impregnated fibres	\\
\enskip\enskip\enskip	03.1.4.04	&	Tyre cord, tubing, conveyor belts	\\
\enskip\enskip\enskip	03.1.4.05	&	Textiles for technical uses, other than nonwovens, yarns, narrow woven fabrics, etc.	\\
\enskip	03.2	&	Articles of apparel and clothing	\\
\enskip\enskip	03.2.1	&	Outer clothing	\\
\enskip\enskip\enskip	03.2.1.01	&	Outer clothing, knitted or crocheted	\\
\enskip\enskip\enskip	03.2.1.02	&	Outer clothing, woven	\\
\enskip\enskip\enskip	03.2.1.03	&	Leather outer clothing	\\
\enskip\enskip\enskip	03.2.1.04	&	Outer clothing made from fur skins	\\
\enskip\enskip\enskip	03.2.1.05	&	Outer clothing made from plastic and rubber	\\
\enskip\enskip	03.2.2	&	Undergarments	\\
\enskip\enskip\enskip	03.2.2.01	&	Corsetry articles such as girdles, corsets, suspenders, etc.	\\
\enskip\enskip\enskip	03.2.2.02	&	Socks, stockings, tights	\\
\enskip\enskip\enskip	03.2.2.03	&	Other knitted or corcheted undergarments such as pyjamas, t-shirts, briefs, etc.	\\
\enskip\enskip\enskip	03.2.2.04	&	Other woven undergarments such as pyjamas, bath robes, briefs, etc.	\\
\enskip\enskip	03.2.3	&	Clothing accessories	\\
\enskip\enskip\enskip	03.2.3.01	&	Hats	\\
\enskip\enskip\enskip	03.2.3.02	&	Gloves	\\
\enskip\enskip\enskip	03.2.3.03	&	Ties	\\
\enskip\enskip\enskip	03.2.3.04	&	Handkerchiefs and scarves	\\
\enskip\enskip\enskip	03.2.3.05	&	Clothing accessories (except hats, gloves, ties, handkerchiefs and scarves)	\\
\enskip	03.3	&	Shoes, parts and accessories	\\
\enskip\enskip	03.3.1	&	Shoes	\\
\enskip\enskip	03.3.2	&	Parts and accessories for shoes	\\
\midrule
	04	&	Paper, articles of paper and and products of the printing industry	\\
\enskip	04.1	&	Basic materials for paper production, such as cellulose and cellulose fibre and paper and carton waste	\\
\enskip	04.2	&	Paper and carton in rolls, strips or sheets	\\
\enskip\enskip	04.2.1	&	Ordinary paper and carton in rolls, strips or sheets	\\
\enskip\enskip	04.2.2	&	Special  paper and carton (painted, coated, stuck, corrugated) in rolls, strips or sheets	\\
\enskip	04.3	&	Goods from paper or carton	\\
\enskip\enskip	04.3.1	&	Office products made from paper or carton	\\
\enskip\enskip	04.3.2	&	Household articles made from paper or carton	\\
\enskip\enskip	04.3.3	&	Technical everyday articles made from paper or carton	\\
\enskip	04.4	&	Products of the printing industry	\\
\enskip\enskip	04.4.1	&	Books, papers, magazines	\\
\enskip\enskip	04.4.2	&	Advertising material, calendars	\\
\enskip\enskip	04.4.3	&	Cards, postage stamps, postcards, prints, etc.	\\
\midrule
	05	&	Leather, rubber, plastics	\\
\enskip	05.1	&	Leather	\\
\enskip\enskip	05.1.1	&	Raw hides and skins	\\
\enskip\enskip	05.1.2	&	Leather and furskins	\\
\enskip\enskip	05.1.3	&	Leather goods (not including clothes, shoes and hats)	\\
\enskip	05.2	&	Rubber	\\
\enskip\enskip	05.2.1	&	Natural rubber and synthetic rubber in its raw state, including basic materials such as latex	\\
\enskip\enskip	05.2.2	&	Semi-manufactures made out of rubber	\\
\enskip\enskip	05.2.3	&	Finished products made of rubber (not including clothes, shoes and hats)	\\
\enskip	05.3	&	Plastics	\\
\enskip\enskip	05.3.1	&	Semi-manufactures made from plastics	\\
\enskip\enskip	05.3.2	&	Finished products made from plastics (not including clothes, shoes and hats)	\\
\midrule
	06	&	Products of the chemical and pharmaceutical industry	\\
\enskip	06.1	&	Chemical raw materials, basic materials and unformed plastics	\\
\enskip\enskip	06.1.1	&	Chemical raw and basic materials	\\
\enskip\enskip\enskip	06.1.1.01	&	Inorganic raw and basic materials	\\
\enskip\enskip\enskip	06.1.1.02	&	Organic raw and basic materials	\\
\enskip\enskip	06.1.2	&	Unformed plastics (primary forms)	\\
\enskip	06.2	&	Chemical end products, vitamins, diagnostic products, including active substances	\\
\enskip\enskip	06.2.1	&	Pharmaceuticals, vitamins, diagnostics (incl. active substances)	\\
\enskip\enskip	06.2.2	&	Agrochemical products	\\
\enskip\enskip\enskip	06.2.2.01	&	Plant protection agents and pesticides	\\
\enskip\enskip\enskip	06.2.2.02	&	Chemical fertilisers	\\
\enskip\enskip	06.2.3	&	Pigments	\\
\enskip\enskip\enskip	06.2.3.01	&	Dyestuffs and pigments	\\
\enskip\enskip\enskip	06.2.3.02	&	Varnishes and paints	\\
\enskip\enskip	06.2.4	&	Essential oils, aromatic and flavouring substances	\\
\enskip\enskip	06.2.5	&	Cosmetics and perfumery products	\\
\enskip\enskip	06.2.6	&	Photochemical products, including unexposed films	\\
\enskip\enskip	06.2.7	&	Commercial tools for the textile industry, leather industry, paper industry and metal industry and cleaning agents	\\
\enskip\enskip	06.2.8	&	Chemical end products such as filler, wax, glue, powder, explosive, solvent	\\
\midrule
	07	&	Stones and earth	\\
\enskip	07.1	&	Mineral raw materials and basic products	\\
\enskip	07.2	&	Goods from stone and cement	\\
\enskip\enskip	07.2.1	&	Building materials from stone or other minerals, including insulating materials	\\
\enskip\enskip	07.2.2	&	Technical articles made from stone or other minerals	\\
\enskip	07.3	&	Ceramic wares	\\
\enskip\enskip	07.3.1	&	Building and sanitary ceramics	\\
\enskip\enskip	07.3.2	&	Ceramic household articles	\\
\enskip	07.4	&	Glass	\\
\enskip\enskip	07.4.1	&	Industrial and commercial glass	\\
\enskip\enskip	07.4.2	&	Glass household articles	\\
\midrule
	08	&	Metals	\\
\enskip	08.1	&	Iron and steel	\\
\enskip\enskip	08.1.1	&	Basic products in primary shapes made from iron and steel	\\
\enskip\enskip	08.1.2	&	Rolled and drawn products made from iron and steel	\\
\enskip\enskip\enskip	08.1.2.01	&	Rolled and drawn products made from iron and non-alloy steel	\\
\enskip\enskip\enskip	08.1.2.02	&	Rolled and drawn products made from alloy steel	\\
\enskip	08.2	&	Non-ferrous metals	\\
\enskip\enskip	08.2.1	&	Copper	\\
\enskip\enskip\enskip	08.2.1.01	&	Basic products in primary shapes made from copper	\\
\enskip\enskip\enskip	08.2.1.02	&	Rolled and drawn products made from copper	\\
\enskip\enskip	08.2.2	&	Aluminium	\\
\enskip\enskip\enskip	08.2.2.01	&	Basic products in primary shapes made from aluminium	\\
\enskip\enskip\enskip	08.2.2.02	&	Rolled and drawn products made from aluminium	\\
\enskip\enskip	08.2.3	&	Non-ferrous metals such as nickel, lead, zinc, tin, etc. (except for copper and aluminium)	\\
\enskip\enskip\enskip	08.2.3.01	&	Basic products in primary shapes made from non-ferrous metals such as nickel, lead, zinc, tin, etc. (except for copper and aluminium)	\\
\enskip\enskip\enskip	08.2.3.02	&	Rolled and drawn products made from non-ferrous metals, such as nickel, lead, zinc, tin, etc. (except for copper and aluminium)	\\
\enskip	08.3	&	Metal goods	\\
\enskip\enskip	08.3.1	&	Metal-based pipes and accessories	\\
\enskip\enskip\enskip	08.3.1.01	&	Pipes and accessories made from iron or steel	\\
\enskip\enskip\enskip	08.3.1.02	&	Pipes and accessories made from non-ferrous metals	\\
\enskip\enskip	08.3.2	&	Metal wire and cable goods	\\
\enskip\enskip	08.3.3	&	Metal containers	\\
\enskip\enskip	08.3.4	&	Metal constructions	\\
\enskip\enskip	08.3.5	&	Machine parts made of metal	\\
\enskip\enskip\enskip	08.3.5.01	&	Housings, fittings, etc. made from metal	\\
\enskip\enskip\enskip	08.3.5.02	&	Chains, springs, fittings, bearings, shafts and the like for machines made of metal	\\
\enskip\enskip	08.3.6	&	Screws, nails, rivets and the like made of metal	\\
\enskip\enskip	08.3.7	&	Mountings, fittings, eyes, locks, etc. made from metal	\\
\enskip\enskip	08.3.8	&	Tools and moulds	\\
\enskip\enskip	08.3.9	&	Metal household appliances, office products made from metal and sanitary ware made from metal	\\
\enskip\enskip\enskip	08.3.9.01	&	Metal household appliances	\\
\enskip\enskip\enskip	08.3.9.02	&	Office products and sanitary ware made from metal	\\
\midrule
	09	&	Machines, appliances, electronics	\\
\enskip	09.1	&	Industrial machinery	\\
\enskip\enskip	09.1.1	&	Non-electrical engines	\\
\enskip\enskip\enskip	09.1.1.01	&	Piston combustion engines	\\
\enskip\enskip\enskip	09.1.1.02	&	Turbines, engines, etc. (gas, water, air, etc.)	\\
\enskip\enskip	09.1.2	&	Construction machines	\\
\enskip\enskip	09.1.3	&	Mechanical engineering	\\
\enskip\enskip\enskip	09.1.3.01	&	Pumps, compressors, ventilators, sprayers and the like	\\
\enskip\enskip\enskip	09.1.3.02	&	Thermo and refrigeration engineering	\\
\enskip\enskip\enskip\enskip	09.1.3.02.01	&	Ventilation, air-conditioning and cooling technology	\\
\enskip\enskip\enskip\enskip	09.1.3.02.02	&	Industrial furnaces	\\
\enskip\enskip\enskip\enskip	09.1.3.02.03	&	Heating and cooling technology appliances such as distilling equipment, vaporizers, dryers, roasting plant, etc.	\\
\enskip\enskip\enskip	09.1.3.03	&	Lifting and conveying apparatus	\\
\enskip\enskip\enskip	09.1.3.04	&	Metal working machines	\\
\enskip\enskip\enskip	09.1.3.05	&	Machines for processing mineral materials	\\
\enskip\enskip\enskip	09.1.3.06	&	Machines for processing rubber and plastics	\\
\enskip\enskip\enskip	09.1.3.07	&	Machines for processing wood, cork, etc.	\\
\enskip\enskip\enskip	09.1.3.08	&	Machine tools other than those for processing metals, mineral materials, rubber, plastics, wood, etc. and forming equipment	\\
\enskip\enskip\enskip	09.1.3.09	&	Hand machine tools	\\
\enskip\enskip\enskip	09.1.3.10	&	Welding machines	\\
\enskip\enskip\enskip	09.1.3.11	&	Machines for the paper and graphic industry	\\
\enskip\enskip\enskip	09.1.3.12	&	Textile machines, including garment machines	\\
\enskip\enskip\enskip	09.1.3.13	&	Machines for food processing	\\
\enskip\enskip\enskip	09.1.3.14	&	Filtering and cleaning machines	\\
\enskip\enskip\enskip	09.1.3.15	&	Packaging and filling machines	\\
\enskip	09.2	&	Agricultural machines	\\
\enskip	09.3	&	Household appliances	\\
\enskip\enskip	09.3.1	&	Consumer electronics	\\
\enskip\enskip	09.3.2	&	Domestic machines	\\
\enskip	09.4	&	Office machines	\\
\enskip	09.5	&	Electrical and electronic industry appliances and devices	\\
\enskip\enskip	09.5.1	&	Power generation appliances and electric motors	\\
\enskip\enskip\enskip	09.5.1.01	&	Alternators, generators, bobbins, transformers, converters, etc.	\\
\enskip\enskip\enskip	09.5.1.02	&	Electric motors and direct current generators	\\
\enskip\enskip	09.5.2	&	Telecommunication appliances	\\
\enskip\enskip	09.5.3	&	Electrical and electronic articles	\\
\enskip\enskip\enskip	09.5.3.01	&	Electrical and electronic components	\\
\enskip\enskip\enskip	09.5.3.02	&	Electrical switching apparatus and cables	\\
\enskip\enskip\enskip\enskip	09.5.3.02.01	&	High and medium voltage switchgear	\\
\enskip\enskip\enskip\enskip	09.5.3.02.02	&	Low voltage switchgear	\\
\enskip\enskip\enskip\enskip	09.5.3.02.03	&	Electrical cable and wire	\\
\enskip\enskip\enskip	09.5.3.03	&	Control, signalling and measuring instruments	\\
\enskip\enskip\enskip	09.5.3.04	&	Electrical machinery and equipment such as magnets, elements, batteries, accumulators, starting equipment, lighting equipment, electrodes, insulators, etc.	\\
\enskip	09.6	&	Military equipment	\\
\midrule
	10	&	Vehicles	\\
\enskip	10.1	&	Road vehicles	\\
\enskip\enskip	10.1.1	&	Road vehicles for transporting people	\\
\enskip\enskip\enskip	10.1.1.01	&	Two-wheeled vehicles	\\
\enskip\enskip\enskip	10.1.1.02	&	Motor cars	\\
\enskip\enskip\enskip	10.1.1.03	&	Buses	\\
\enskip\enskip	10.1.2	&	Road vehicles for transporting goods	\\
\enskip\enskip\enskip	10.1.2.01	&	Lorries	\\
\enskip\enskip\enskip	10.1.2.02	&	Tractors	\\
\enskip\enskip	10.1.3	&	Special road vehicles, such as utility vehicles, caravans, etc.	\\
\enskip\enskip	10.1.4	&	Spare parts for road vehicles	\\
\enskip	10.2	&	Railed vehicles	\\
\enskip\enskip	10.2.1	&	Rolling stock such as locomotives, railed vehicles for maintenance, railway carriages, etc.	\\
\enskip\enskip	10.2.2	&	Spare parts for railed vehicles and railway facilities	\\
\enskip	10.3	&	Air- and spacecraft	\\
\enskip\enskip	10.3.1	&	Aircraft	\\
\enskip\enskip	10.3.2	&	Spare parts for the aircraft and aerospace industry and ground facilities	\\
\enskip	10.4	&	Watercraft	\\
\midrule
	11	&	Precision instruments, clocks and watches and jewellery	\\
\enskip	11.1	&	Precision instruments and equipment	\\
\enskip\enskip	11.1.1	&	Optical instruments	\\
\enskip\enskip\enskip	11.1.1.01	&	Lenses, prisms, spectacles and the like	\\
\enskip\enskip\enskip	11.1.1.02	&	Microscopes, telescopes, laser, etc.	\\
\enskip\enskip\enskip	11.1.1.03	&	Photographic and filming equipment	\\
\enskip\enskip	11.1.2	&	Surveying instruments	\\
\enskip\enskip	11.1.3	&	Medical instruments and equipment	\\
\enskip\enskip	11.1.4	&	Mechanical measuring, testing and regulating equipment	\\
\enskip	11.2	&	Watches	\\
\enskip\enskip	11.2.1	&	Small watches	\\
\enskip\enskip\enskip	11.2.1.01	&	Small watches, electrically operated	\\
\enskip\enskip\enskip	11.2.1.02	&	Small watches, not electrically operated	\\
\enskip\enskip	11.2.2	&	Big clocks	\\
\enskip\enskip	11.2.3	&	Time clocks and time switches	\\
\enskip\enskip	11.2.4	&	Clock or watch parts	\\
\enskip\enskip\enskip	11.2.4.01	&	Movements	\\
\enskip\enskip\enskip	11.2.4.02	&	Clock cases	\\
\enskip\enskip\enskip	11.2.4.03	&	Clock and watch parts, other than movements and clock cases	\\
\enskip	11.3	&	Jewellery and household goods made from precious metals	\\
\enskip\enskip	11.3.1	&	Jewellery	\\
\enskip\enskip	11.3.2	&	Articles of everyday use made from precious metals	\\
\midrule
	12	&	Various goods such as music instruments, home furnishings, toys, sports equipment, etc.	\\
\enskip	12.1	&	Exposed film	\\
\enskip	12.2	&	Music instruments	\\
\enskip	12.3	&	Home furnishings	\\
\enskip\enskip	12.3.1	&	Furniture	\\
\enskip\enskip	12.3.2	&	Small items of carpenters, lamps and suchlike	\\
\enskip	12.4	&	Toys and sports equipment	\\
\enskip\enskip	12.4.1	&	Toys	\\
\enskip\enskip	12.4.2	&	Sports equipment	\\
\enskip	12.5	&	Stationery goods	\\
\enskip	12.6	&	Various goods such as umbrellas, neon signs, festive articles, brushes, lighters, pipes, etc.	\\
\midrule
	13	&	Precious metals, precious and semi-precious stones	\\
\enskip	13.1	&	Precious and semi-precious stones	\\
\enskip	13.2	&	Precious metals (including gold and silver bars from 1.1.2012)	\\
\midrule
	14	&	Works of art and antiques	\\
\enskip	14.1	&	Works of art	\\
\enskip	14.2	&	Antiques and collectors' items	\\
\end{longtable}
\end{small}




\end{document}